% $Header: svn+ssh://andre@crapman/home/junda/repos/andre/school/trunk/m/Chapter/stat/probability/chapter/prob.intro.tex 221 2021-02-13 18:18:09Z andre $

% \part{Kombi}
\section{Wahrscheinlichkeitsrechnung I}
\subsection*{Intro}
%= = = = = = = = = = = = = = = = = = = = = = = = = = = = = = = = = = = = = = =
%= = = = = = = = = = = = = = = = = = = = = = = = = = = = = = = = = = = = = = =
% \iffalse
\begin{frame}
  \frametitle{Intro}
%  \framesubtitle{H\"oren}
  %%= = = = = = = = = = = = = = = = = = = = = = = = = = = = = = = = = = = = = =
  Todo
  what ist
  %%= = = = = = = = = = = = = = = = = = = = = = = = = = = = = = = = = = = = = =

\end{frame}
%==============================================================================

% % \section{Intro}
% \subsection*{Intro}
% %= = = = = = = = = = = = = = = = = = = = = = = = = = = = = = = = = = = = = = =
% %= = = = = = = = = = = = = = = = = = = = = = = = = = = = = = = = = = = = = = =
% % \iffalse
% \begin{frame}
%   \frametitle{Modelle}
%   \framesubtitle{Settings}
% %  \framesubtitle{Settings I}
%   %%= = = = = = = = = = = = = = = = = = = = = = = = = = = = = = = = = = = = = =
%
% \begin{block}{Settings I}
%   Die Kugeln werden nach der Ziehung
%   \begin{description}
%     \item [wieder zur\"uckgelegt]: der Ball kann daher \"ofter
%         gezogen werden, oder
%     \item [zur Seite gelegt]: dann wird der Ball maximal
%   einmal gezogen.
%   \end{description}
% \end{block}
%
% \pause
% \begin{block}{Settings II}
%   In der Urne befinden sich Kugeln, die
%   \begin{description}
%     \item [unterscheidbar sind]: jeder Ball ist ein Unikat (Farbe, Nummer,
%         Material, \dots)
%     \item [nicht unterscheidbar sind]: es gibt B\"alle, die die gleichen
%         Eigenschaften haben (z.~B. drei blaue und zwei rote B\"alle)
%   \end{description}
% \end{block}
%
% \end{frame}
% %==============================================================================
%
% % \section{Intro}
% % \subsection*{Modelle}
% %= = = = = = = = = = = = = = = = = = = = = = = = = = = = = = = = = = = = = = =
% %= = = = = = = = = = = = = = = = = = = = = = = = = = = = = = = = = = = = = = =
% % \iffalse
% \begin{frame}
%   \frametitle{Modelle}
%   \framesubtitle{Settings}
%   %%= = = = = = = = = = = = = = = = = = = = = = = = = = = = = = = = = = = = = =
%
% \begin{block}{Settings III}
%   Aus einer Urne mit $n$ Kugeln werden $k$ B\"alle gezogen.
%   \begin{description}
%     \item [$k=n$]: Permutationen
%     \item [$k<n$]: Variationen (alle B\"alle sind unterscheidbar)
%       und Kombinationen (B\"alle mit gleichen Eigenschaften)
%   \end{description}
% \end{block}
%
% \pause
% Aus wikipedia {\tiny(\url{https://de.wikipedia.org/wiki/Abz\%C3\%A4hlende_Kombinatorik\#Begriffsabgrenzungen})}:\\
% % https://de.wikipedia.org/wiki/Abz%C3%A4hlende_Kombinatorik#Begriffsabgrenzungen
% {\tiny
% Aufgrund der Vielfalt der Herangehensweisen sind die Schreibweisen und
% Begrifflichkeiten im Bereich der Kombinatorik leider oft recht uneinheitlich.
% Zwar bezeichnen übereinstimmend alle Autoren die Vertauschung der Reihenfolge
% einer Menge von $n$ unterscheidbaren Elementen als Permutation. W\"ahlt man
% dagegen von diesen $n$ Elementen nur $k < n$  Elemente aus, deren Reihenfolge
% man anschließend vertauscht, bezeichnen viele Autoren das nun als Variation,
% geordnete Stichprobe bzw. Kombination mit Berücksichtigung der Reihenfolge,
% andere dagegen (namentlich im englischsprachigen Raum) weiter als Permutation.
% L\"asst man schließlich in einer solchen Auswahl von Elementen deren
% Reihenfolge au\ss{}er Acht, wird solch eine Auswahl nun für gew\"ohnlich
% ungeordnete Stichprobe, Kombination ohne Berücksichtigung der Reihenfolge
% oder einfach nur Kombination genannt. Kombinationen sind also, sofern
% nichts weiter zu ihnen gesagt wird, in der Regel ungeordnet, Permutationen
% und/oder Variationen dagegen geordnet, wobei die Frage, ob man Permutationen
% als Sonderf\"alle von Variationen (oder umgekehrt) betrachtet, gegebenenfalls
% von Autor zu Autor unterschiedlich beantwortet wird.\\
% Alles in allem gibt es also zunächst einmal drei (oder auch nur zwei)
% verschiedene Fragestellungen, die ihrerseits noch einmal danach unterteilt
% werden, ob es unter den ausgewählten Elementen auch Wiederholungen gleicher
% Elemente geben darf oder nicht. Ist ersteres der Fall, spricht man von
% Kombinationen, Variationen oder Permutationen mit Wiederholung, andernfalls
% solchen ohne Wiederholung. Stellt man sich schließlich vor, dass die
% ausgew\"ahlten Elemente dabei einer Urne oder Ähnlichem entnommen werden,
% wird dementsprechend auch von Stichproben mit oder ohne Zur\"ucklegen
% gesprochen.\\
% }
%
% \end{frame}
% %==============================================================================
